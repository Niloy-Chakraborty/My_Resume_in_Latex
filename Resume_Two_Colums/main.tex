%%%%%%%%%%%%%%%%%

% \documentclass[10pt,a4paper,academicons]{altacv}


\documentclass[10pt,a4paper,ragged2e]{classfile}


% Change the page layout if you need to
\geometry{left=1cm,right=10cm,marginparwidth=7.8cm,marginparsep=1.2cm,top=1.25cm,bottom=1.5cm}

% Change the font if you want to, depending on whether
% you're using pdflatex or xelatex/lualatex
\ifxetexorluatex
  % If using xelatex or lualatex:
  \setmainfont{Carlito}
\else
  % If using pdflatex:
  \usepackage[utf8]{inputenc}
  \usepackage[T1]{fontenc}
  \usepackage[default]{lato}
\fi
\usepackage{hyperref}
\hypersetup{
    colorlinks=true,
    linkcolor=blue,
    filecolor=magenta,      
    urlcolor=cyan,
}
% Change the colours if you want to
\definecolor{Sky}{HTML}{2d7dad}
\definecolor{SlateGrey}{HTML}{2E2E2E}
\definecolor{LightGrey}{HTML}{666666}
\definecolor{deepsky}{HTML}{074870}
%\colorlet{heading}{VividPurple}
\colorlet{accent}{Sky}
\colorlet{emphasis}{SlateGrey}
\colorlet{body}{LightGrey}
\colorlet{heading}{deepsky}
% Change the bullets for itemize and rating marker
% for \cvskill if you want to
\renewcommand{\itemmarker}{{\small\textbullet}}
\renewcommand{\ratingmarker}{\faCircle}

%% sample.bib contains your publications
\addbibresource{sample.bib}

%% Background Beamer
\usepackage{eso-pic}
\newcommand\BackgroundPic{%
\put(0,0){%
\parbox[b][\paperheight]{\paperwidth}{%

\includegraphics[width=22.0cm,height=5.0cm]{background.png}%
\vfill
}}}

\usepackage{enumitem,xcolor}
\usepackage{xcolor}


% Hyperlink ICON
% Compile with LuaLaTeX 
%\documentclass{article}
\usepackage{fontawesome}
\usepackage[hidelinks]{hyperref}

% Redefinition, symbol included in link:
\let\orighref\href
\renewcommand{\href}[2]{\orighref{#1}{#2\,\faExternalLink}}







\begin{document}
\AddToShipoutPicture*{\BackgroundPic}

\name{ NILOY CHAKRABORTY}
\tagline{Master's Student At University of Stuttgart}
% Cropped to square from https://en.wikipedia.org/wiki/Marissa_Mayer#/media/File:Marissa_Mayer_May_2014_(cropped).jpg, CC-BY 2.0
\photo{3.5cm}{niloy.png}
\personalinfo{%
  % Not all of these are required!
  % You can add your own with \printinfo{symbol}{detail}
   \phone{+49-15171684000 }
  \mailaddress{Allmandring-24-c, Zi-20802}
  \location{ 70569 Stuttgart}
  \email{chakrabortyniloy2018@gmail.com }
 \github{https://github.com/Niloy-Chakraborty} 
}

%% Make the header extend all the way to the right, if you want.
\begin{fullwidth}
\makecvheader
\end{fullwidth}
%% Depending on your tastes, you may want to make fonts of itemize environments slightly smaller
\AtBeginEnvironment{itemize}{\small}

%% Provide the file name containing the sidebar contents as an optional parameter to \cvsection.
%% You can always just use \marginpar{...} if you do
%% not need to align the top of the contents to any
%% \cvsection title in the "main" bar.

\cvsection[page1sidebar]{Experience}
\cvevent{Intern in Predictive Analytics}{Robert Bosch GmbH}{March 2020 --Present}{Stuttgart ,Germany }
\begin{itemize}[label=\textcolor{blue}{\textbullet}]
\item Log data parsing and Process Mining of the in-house Software (myMBR) to bring insights from data
\item User Journey analysis through Process Analytics and finding the bottlenecks
\item Creation of Roles, and KPIs for myMBR, Analysis of Monthly Business data using Tableau and MS Power BI
\item Development of Purchase Order Denial Model using Machine
Learning Techniques.



\divider

\end{itemize}

\cvevent{Student Assistant}{Fraunhofer IAO}{Feb 2019 --Dec,2019}{Stuttgart ,Germany }
\begin{itemize}[label=\textcolor{blue}{\textbullet}]
\item Time-series data analysis of Smart Meter data for a smart energy system project.
\item Time-series stream clustering using different Machine Learning Techniques for Smart Energy Systems.
\item Building a Web Application for showing the results.

\divider

\end{itemize}

\cvevent{Scientific HiWi}{IAAS-University of Stuttgart}{July 2019 --Dec,2019}{Stuttgart ,Germany }
\begin{itemize}[label=\textcolor{blue}{\textbullet}]
\item Building data flow pipeline for smart data-centers.
\item Implemented real-time data monitoring using InfluxDB.
\item Prediction of CPU load for adopting smart decisions.

\divider

\end{itemize}

\cvevent{Assistant System Engineer}{Tata Consultancy Services Limited}{Oct 2016 --Sept 2018}{Hyderabad ,India }
\begin{itemize}[label=\textcolor{blue}{\textbullet}]
\item Web Application development using HTML, CSS, JavaScript in front end and python based frameworks in the back end.
\item Developed different Business Solutions and Proof of Concepts using Machine Learning and Deep Learning Techniques.
\item \textbf{PoCs Developed: }IP Network Anomaly Detection, Cognitive KPI
Prediction, Network Traffic Classification.
\end{itemize}


\divider


\cvevent{Undergraduate Research Intern}{National Institute of Technology, Sikkim}{April 2014 -- June 2016}{Sikkim ,India }
\begin{itemize}[label=\textcolor{blue}{\textbullet}]
\item Designed Photonic Crystal Antenna between the range of 200THz and 210THz using CST studio

\end{itemize}
\divider

\newpage

\cvsection[page2sidebar]{Self-Paced Projects}

\textbf{Model Based Test Driven Development}

\begin{itemize}[label=\textcolor{blue}{\textbullet}]
\item An approach for developing faster Regression test suites and also in executing multiple test cases with less manual intervention 
\item  It uses Finite State Machines for visual representation of test cases.
\item It also has features like tree visualization of the yang files, python code editor for writing test cases , syntax checker, graph editor etc.
\end{itemize}


\textbf{My Smart Home Garden}
\href{https://github.com/Niloy-Chakraborty/Smart-Gardening}{}
\begin{itemize}[label=\textcolor{blue}{\textbullet}]
\item A gardening system (Watering, Lighting, and Plant Identification) was automated, which could be operated using Telegram bot.
\item Cloud based data storage through Message Queuing for sensor and weather data, which could also be visualized in Telegram bot.
\item An AI planning model was developed for automatic watering and lighting. 
\item CNN based plant name and health identification models were developed.
\end{itemize}

\textbf{Real Time Data Monitoring System}
\href{https://github.com/Niloy-Chakraborty/Real-Time-Data-Monitoring-System}{}

\begin{itemize}[label=\textcolor{blue}{\textbullet}]
\item This project shows an end-to-end data flow pipeline via RabbitMQ and InfluxDB for sensor/ network KPI data.
\item The data can be visualized via Chronograf in Real-Time for further analysis
\end{itemize}

\textbf{Website Saliency Prediction}
\href{https://github.com/Niloy-Chakraborty/Webpage_Saliency_Prediction}{}
\begin{itemize}[label=\textcolor{blue}{\textbullet}]
\item A study of web-page saliency using FiWi dataset (only 149 images). 
\item A VGG-16 based Fully Convolutional Neural Network with skip connection was implemented to predict the Human attention map of the Webpages.
\item A 2 stage Transfer Learning process was used to compensate the lack of training data and the results were compared with the top saliency models.

\end{itemize}

\textbf{Time Series Stream Clustering for Smart Meter Dataset}
\href{https://github.com/Niloy-Chakraborty/Time-Series_Clustering_on_London_Smart_Meter_Dataset}{}
\begin{itemize}[label=\textcolor{blue}{\textbullet}]
\item This project implemented a prototype of the time-series clustering of the London Smart Meter Dataset using various Machine Learning algorithms, like K-means, Hierarchical clustering, and Auto encoder.
\item The clustered groups were visualized for better insights of the data.
\end{itemize}

\textbf{Pima Indian Diabetes Classification}
\href{https://github.com/Niloy-Chakraborty/Pima_Indian_Diabetes_Classification}{}
\begin{itemize}[label=\textcolor{blue}{\textbullet}]
\item In this project, the Pima Indian Diabetes dataset has been
classified using the Random Forest classifier. 
\item Although the initial accuracy was 74\%, it has been increased to 79\% using Feature Selection technique. 

\end{itemize}



\cvsection{Honors \& Awards}
\begin{itemize}[label=\textcolor{blue}{\textbullet}]
    \item  Won various awards (On the Spot, Special Initiative award, etc.) during my Full time job for my contribution in the R\&D
    \item Consistently Ranked among top 5\% during Undergrad .
    \item Honoured with "VIDYASHREE" titled award for outstanding performance in the 10th Grade, from Amul Group, India
    \item Awarded for serving as Placement Coordinator during Undergrad.
    \item Received Scholarship from All India Council for Technical Education for National Level Training Program in Telecom. Infrastructure.


\end{itemize}


\end{document}

